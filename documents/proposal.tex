%%%%%%%%%%%%%%%%%%%%%%%%%%%%%%%%%%%%%%%%%
% University Assignment Title Page 
% LaTeX Template
% Version 1.0 (27/12/12)
%
% This template has been downloaded from:
% http://www.LaTeXTemplates.com
%
% Original author:
% WikiBooks (http://en.wikibooks.org/wiki/LaTeX/Title_Creation)
%
% License:
% CC BY-NC-SA 3.0 (http://creativecommons.org/licenses/by-nc-sa/3.0/)
% 
% Instructions for using this template:
% This title page is capable of being compiled as is. This is not useful for 
% including it in another document. To do this, you have two options: 
%
% 1) Copy/paste everything between \begin{document} and \end{document} 
% starting at \begin{titlepage} and paste this into another LaTeX file where you 
% want your title page.
% OR
% 2) Remove everything outside the \begin{titlepage} and \end{titlepage} and 
% move this file to the same directory as the LaTeX file you wish to add it to. 
% Then add \input{./title_page_1.tex} to your LaTeX file where you want your
% title page.
%
%%%%%%%%%%%%%%%%%%%%%%%%%%%%%%%%%%%%%%%%%
%\title{Title page with logo}
%----------------------------------------------------------------------------------------
%	PACKAGES AND OTHER DOCUMENT CONFIGURATIONS
%----------------------------------------------------------------------------------------

\documentclass[11pt]{article}
\usepackage[english]{babel}
\usepackage[utf8x]{inputenc}
\usepackage{amsmath}
\usepackage{graphicx}
\usepackage[colorinlistoftodos]{todonotes}
\usepackage{enumitem}
\usepackage{listings}
\usepackage{filecontents}
\usepackage{hyperref}
\usepackage{doi}
\usepackage{natbib}
\usepackage{verbatim}
\usepackage{eurosym}
\usepackage[export]{adjustbox}


\begin{document}

\begin{titlepage}

\newcommand{\HRule}{\rule{\linewidth}{0.5mm}} % Defines a new command for the horizontal lines, change thickness here

\center % Center everything on the page
 
%----------------------------------------------------------------------------------------
%	HEADING SECTIONS
%----------------------------------------------------------------------------------------

\textsc{\LARGE University of Groningen}\\[1.5cm] % Name of your university/college
\textsc{\Large Multi-Agent Systems}\\[0.5cm] % Major heading such as course name
%\textsc{\large Assignment 1}\\[0.5cm] % Minor heading such as course title

%----------------------------------------------------------------------------------------
%	TITLE SECTION
%----------------------------------------------------------------------------------------

\HRule \\[0.4cm]
{ \huge \bfseries What Effect Does the Introduction of Broadcasting Agents in the DIAL Framework Have on Opinion Radicalisation and Group Segregation?}\\[0.4cm] % Title of your document
\HRule \\[1.5cm]
 
%----------------------------------------------------------------------------------------
%	AUTHOR SECTION
%----------------------------------------------------------------------------------------

\begin{minipage}{0.4\textwidth}
\begin{flushleft} \large
\emph{Authors:}\\
% Name \textsc{Surname} \textit{(s1234567)} \\
\textsc{Akbari, Mortaza;  Gassilloud, Andreas; Groenendaal, Georg; Stockert, Laurids}

\end{flushleft}
\end{minipage}
~
\begin{minipage}{0.4\textwidth}
\begin{flushright} \large
\emph{Lecturer:} \\
dr. Malvin \textsc{Gattinger} \\
\end{flushright}
\end{minipage}\\[2cm]

% If you don't want a supervisor, uncomment the two lines below and remove the section above
%\Large \emph{Author:}\\
%John \textsc{Smith}\\[3cm] % Your name

%----------------------------------------------------------------------------------------
%	DATE SECTION
%----------------------------------------------------------------------------------------

{\large \today}\\[2cm] % Date, change the \today to a set date if you want to be precise

%----------------------------------------------------------------------------------------
%	LOGO SECTION
%----------------------------------------------------------------------------------------

% \includegraphics[width=50px, keepaspectratio]{logo_rug.jpg}\\[1cm] % Include a department/university logo - this will require the graphicx package
 
%----------------------------------------------------------------------------------------





\vfill % Fill the rest of the page with whitespace

\end{titlepage}

\section{Proposal}
\cite{dykstra2013} used a multi-agent system to study opinion dynamics and social influence. The agents' behaviour is informed by the theory of reasoned action and social judgement theory. This covers the composition of attitudes, i.e. the sum of weighted beliefs, as well as the circumstances under which an agent changes its belief. In this environment, agents compete with each other in dialogues for reputation points. \cite{dykstra2013} mainly evaluated the emergent behaviour in terms of social outcomes. What they observed was intra-group radicalisation of opinion. Our goal is to introduce a different type of agent, a so-called media agent/broadcasting agent, and study if their presence has an impact on opinion radicalisation or group segregation. In particular, it may be of interest to study the effects of having media agents represent the diversity of opinion in the population versus not having that. Our idea of a media agent is thus far rather vague, but includes some of the following characteristics:
\begin{itemize}
  \item Normal agents can move in physical space (representing the proximity of social relations). We want to either propose that media agents have a high inertia or do not move at all. The implicit assumption here is that media outlets remain consistent in their political beliefs, but people do not.
  \item Media agents have a higher reputation than normal agents. 
  \item Media agents cannot be attacked / challenged by regular agents.
  \item A media agent's reputation is a function of the number of regular agents supporting its views.
\end{itemize}

\bibliographystyle{plainnat}
\bibliography{lib.bib}

\end{document}