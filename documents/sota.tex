%%%%%%%%%%%%%%%%%%%%%%%%%%%%%%%%%%%%%%%%%
% University Assignment Title Page 
% LaTeX Template
% Version 1.0 (27/12/12)
%
% This template has been downloaded from:
% http://www.LaTeXTemplates.com
%
% Original author:
% WikiBooks (http://en.wikibooks.org/wiki/LaTeX/Title_Creation)
%
% License:
% CC BY-NC-SA 3.0 (http://creativecommons.org/licenses/by-nc-sa/3.0/)
% 
% Instructions for using this template:
% This title page is capable of being compiled as is. This is not useful for 
% including it in another document. To do this, you have two options: 
%
% 1) Copy/paste everything between \begin{document} and \end{document} 
% starting at \begin{titlepage} and paste this into another LaTeX file where you 
% want your title page.
% OR
% 2) Remove everything outside the \begin{titlepage} and \end{titlepage} and 
% move this file to the same directory as the LaTeX file you wish to add it to. 
% Then add \input{./title_page_1.tex} to your LaTeX file where you want your
% title page.
%
%%%%%%%%%%%%%%%%%%%%%%%%%%%%%%%%%%%%%%%%%
%\title{Title page with logo}
%----------------------------------------------------------------------------------------
%	PACKAGES AND OTHER DOCUMENT CONFIGURATIONS
%----------------------------------------------------------------------------------------

\documentclass[11pt]{article}
\usepackage[english]{babel}
\usepackage[utf8x]{inputenc}
\usepackage{amsmath}
\usepackage{graphicx}
\usepackage[colorinlistoftodos]{todonotes}
\usepackage{enumitem}
\usepackage{listings}
\usepackage{filecontents}
\usepackage{hyperref}
\usepackage{doi}
\usepackage{natbib}
\usepackage{verbatim}
\usepackage{eurosym}
\usepackage[export]{adjustbox}


\begin{document}

\begin{titlepage}

\newcommand{\HRule}{\rule{\linewidth}{0.5mm}} % Defines a new command for the horizontal lines, change thickness here

\center % Center everything on the page
 
%----------------------------------------------------------------------------------------
%	HEADING SECTIONS
%----------------------------------------------------------------------------------------

\textsc{\LARGE University of Groningen}\\[1.5cm] % Name of your university/college
\textsc{\Large Multi-Agent Systems}\\[0.5cm] % Major heading such as course name
%\textsc{\large Assignment 1}\\[0.5cm] % Minor heading such as course title

%----------------------------------------------------------------------------------------
%	TITLE SECTION
%----------------------------------------------------------------------------------------

\HRule \\[0.4cm]
{ \huge \bfseries State Of the Art Analysis\\[0.4cm] % Title of your document
\HRule \\[1.5cm]
 
%----------------------------------------------------------------------------------------
%	AUTHOR SECTION
%----------------------------------------------------------------------------------------

\begin{minipage}{0.4\textwidth}
\begin{flushleft} \large
\emph{Authors:}\\
% Name \textsc{Surname} \textit{(s1234567)} \\
\textsc{Akbari, Mortaza;  Gassilloud, Andreas; Groenendaal, Georg; Stockert, Laurids}

\end{flushleft}
\end{minipage}
~
\begin{minipage}{0.4\textwidth}
\begin{flushright} \large
\emph{Lecturer:} \\
dr. Malvin \textsc{Gattinger} \\
\end{flushright}
\end{minipage}\\[2cm]

% If you don't want a supervisor, uncomment the two lines below and remove the section above
%\Large \emph{Author:}\\
%John \textsc{Smith}\\[3cm] % Your name

%----------------------------------------------------------------------------------------
%	DATE SECTION
%----------------------------------------------------------------------------------------

{\large \today}\\[2cm] % Date, change the \today to a set date if you want to be precise

%----------------------------------------------------------------------------------------
%	LOGO SECTION
%----------------------------------------------------------------------------------------

% \includegraphics[width=50px, keepaspectratio]{logo_rug.jpg}\\[1cm] % Include a department/university logo - this will require the graphicx package
 
%----------------------------------------------------------------------------------------





\vfill % Fill the rest of the page with whitespace

\end{titlepage}

% \section{Proposal}
% \cite{dykstra2013} used a multi-agent system to study opinion dynamics and social influence. The agents' behaviour is informed by the theory of reasoned action and social judgement theory. This covers the composition of attitudes, i.e. the sum of weighted beliefs, as well as the circumstances under which an agent changes its belief. In this environment, agents compete with each other in dialogues for reputation points. \cite{dykstra2013} mainly evaluated the emergent behaviour in terms of social outcomes. What they observed was intra-group radicalisation of opinion. Our goal is to introduce a different type of agent, a so-called media agent/broadcasting agent, and study if their presence has an impact on opinion radicalisation or group segregation. In particular, it may be of interest to study the effects of having media agents represent the diversity of opinion in the population versus not having that. Our idea of a media agent is thus far rather vague, but includes some of the following characteristics:
% \begin{itemize}
%   \item Normal agents can move in physical space (representing the proximity of social relations). We want to either propose that media agents have a high inertia or do not move at all. The implicit assumption here is that media outlets remain consistent in their political beliefs, but people do not.
%   \item Media agents have a higher reputation than normal agents. 
%   \item Media agents cannot be attacked / challenged by regular agents.
%   \item A media agent's reputation is a function of the number of regular agents supporting its views.
% \end{itemize}


% \documentclass{article}
% \usepackage[utf8]{inputenc}
% \usepackage[english]{babel}


% \title{Draft, Version 1 - State of the art analysis}
% \author{Laurids}
% \date{September 2020}

% \begin{document}

% \maketitle

\section{What is the problem addressed?}

Humans are social creatures and thus they participate in social networks. These networks involve an interaction of large numbers of people who all have opinions. In part, these opinions are formed through communication and information exchange with other people. An emergent property of these social networks is group-formation and, as a consequence, group segregation and opinion radicalisation. Due to the high number of social interactions, chain reactions or cascades of events are not unusual, which makes it too difficult to predict opinion dynamics using purely analytic methods. Hence, agent-based modeling can cast some light on the evolution of social outcomes in social networks.

\section{What is the state of the art concerning this problem?}
In 2013, Dykstra et al. put forward a multi-agent system framework called DIAL to model opinion dynamics in social networks. DIAL (short for dialogue) is a space in which agents compete for reputation points by making statements and arguing with each other. A notable difference compared to previous work in this field is the inclusion of theory of mind, as agents are able to reason about other agents' beliefs. DIAL is a simple model that is theoretically informed by theories from social psychology, including social judgement theory \cite{sherif&hovland1961} and the theory of reasoned action{}.

\section{What is the new idea for addressing this problem?}
We argue that \cite{dykstra2013} model opinion dynamics based on incomplete assumptions. In particular, people derive a major part of their information ,and ergo opinion formation, from large broadcasters such as television programs, newspapers, and online blogs or magazines. These organisations differ from normal people in several ways. Firstly, they usually have a much larger following that does not depend on social relations but rather political affiliation. Secondly, their views tend to remain quite consistent, whereas individual views are often subjected to large volatility {REFERENCE}. And thirdly, and maybe most importantly, the diversity in media opinions is not necessarily a reflection of a population's diversity of opinion, even in societies that explicitly value free speech {REFERENCE}. As media has an undeniable influence on a population's opinion formation, we propose to extend the DIAL framework by introducing so-called media agents / broadcasting agents and investigate the consequences on social outcomes. 

\section{What are the results? (expected or established)}
As their central conclusion, \cite{dykstra2013}. showed that radicalisation is either prevented or faciliated depending on whether agents conform to information or norms, respectively. In other words, individual agents exhibit opinion radicalisation if their reputation is largely determined by assimilation to their normative environment, not by winning debates.


\section{What is the relevance of this work?}
Modeling opinion dynamics in social networks can help us better understand phenomena like online radicalisation, the filter bubble effect or even major political upheavals such as the Arab spring. In particular, improving the comprehension of what effect large media outlets have on individual and group opinion can help to promote an environment of conducive compromises rather than polarisation and divisiveness. 


\bibliographystyle{plainnat}
\bibliography{lib.bib}

\end{document}